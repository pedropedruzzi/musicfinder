\chapter{Contribui��es}

O prot�tipo desenvolvido ao longo deste trabalho tocou diversas �reas de pesquisa, desde a an�lise de sinal, at� a adaptatividade. Em sua maioria, foram utilizadas t�cnicas previamente criadas e desenvolvidas em trabalhos anteriores, por�m o foco do trabalho:

\begin{quote}
\emph{Uso de aut�matos adaptativos para reconhecimento de padr�es musicais}
\end{quote}

apresentou uma abordagem diferente para um tema de grande import�ncia: o estabelecimento de um algoritmo de similaridade entre conte�dos de �udio.

A t�cnica descrita passa a fazer parte de um leque de possibilidades para utiliza��o dentro de um sistema de busca de �udio por conte�do mais complexo. O m�todo descrito traz a aplica��o de uma t�cnica adaptativa como ferramenta de reconhecimento de padr�es mel�dicos e avalia��o de similaridade. Apesar de ainda haver a necessidade da determina��o dos pesos ideiais para se atingir um crit�rio adequado de c�lculo de similaridade, o trabalho prov� a base para tal utiliza��o.

T�cnicas adaptativas j� foram usadas em diversos trabalhos com o fim de reconhecer padr�es, por�m pouco utilizadas no dom�nio de �udio, esse trabalho apresenta uma poss�vel aplica��o de uma t�cnica adaptativa neste contexto, no caso, um aut�mato adaptativo, mostrando que pode-se obter bons resultados.